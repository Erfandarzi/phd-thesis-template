\documentclass{dissertation}

%% Turn off page numbering for the propositions and make the margins on both
%% sides equal and symmetrical.
\geometry{twoside=false}
\pagestyle{empty}

\RequirePackage{unicode-math}

\setmainfont[Path = fonts/libertinus/, ItalicFont=libertinusserif-italic.otf, BoldFont=libertinusserif-bold.otf, BoldItalicFont=libertinusserif-bolditalic.otf]{libertinusserif-regular.otf}
\setsansfont[Path = fonts/libertinus/, BoldFont=libertinussans-bold.otf, ItalicFont=libertinussans-italic.otf]{libertinussans-regular.otf}
\setmathfont[Path = fonts/libertinus/]{libertinusmath-regular.otf}
\setmonofont[Scale=MatchLowercase]{inconsolata}
%% The default style for text is Tahoma (sans-serif).
%\renewcommand*\familydefault{\rmfamily}
    
\begin{document}

%% Specify the title and author of the thesis. This information will be used on
%% both the English and Dutch side and in the metadata of the final PDF.
\title{An Empirical Evaluation of Feedback-Driven Development}
\author{Moritz Marc}{Beller}

\begin{center}

{\Large\titlefont\bfseries Propositions}

\bigskip

accompanying the dissertation

\bigskip

%% Print the title.
{\makeatletter
\titlestyle\bfseries\large\@title
\makeatother}

%% Print the optional subtitle.
{\makeatletter
\ifx\@subtitle\undefined\else
    \titlefont\titleshape\@subtitle
\fi
\makeatother}

\bigskip

by

\bigskip

%% Print the full name of the author.
\makeatletter
{\large\titlefont\bfseries\@firstname\ {\titleshape\@lastname}}
\makeatother

\end{center}

\bigskip
\bigskip

\begin{enumerate}
\item Research on testing should happen in a language with state-of-the-art testing practices such
  as Ruby, rather than Java. [This thesis]
\item Once we have uncovered a new phenomenon like the Last Line Effect, we can never research
  it again from the same standpoint because we have interfered with it. [This thesis]
\item To make case studies in Software Engineering relevant beyond their publishing date, we need
  self-updating papers.
\item The fundamental problem of Software Engineering, in contrast to most other science
  fields, is its inherent lack of generality in the goals different projects have.
\item Reviewers who write comments à la ``this finding is not interesting or
  surprising'' should read crime novels instead of scientific papers.
\item Most citations are superficial and do not embed or work with the cited paper.
\item Reviewers are more likely to reveal their identity to authors in the case of a positive
  review, creating a dependency of gratitude on the authors. This gives rise to conflicts of
  interest.
\item The current academic model does not incentivize true collaboration.
\item The productivity of PhD students is directly correlated with how well-intentioned their
  supervisors are.
\item Most papers contain at least one statistical result that is wrong. [This
  thesis]
\end{enumerate}

\bigskip
\bigskip

%% Apart from the name and title of the supervisor, the following text is
%% dictated by the promotieregelement.
\begin{center}
These propositions are regarded as opposable and defendable, and have been approved as such by the
promotors prof.\ dr.\ A.\ van Deursen, dr.\ A.\ Zaidman, and dr.\ G.\ Gousios.
\end{center}

%% \clearpage
%% {\selectlanguage{dutch}

%% \begin{center}

%% {\Large\titlefont\bfseries Stellingen}

%% \bigskip

%% behorende bij het proefschrift

%% \bigskip

%% %% Print the title.
%% {\makeatletter
%% \titlestyle\bfseries\large\@title
%% \makeatother}

%% %% Print the optional subtitle.
%% {\makeatletter
%% \ifx\@subtitle\undefined\else
%%     \titlefont\titleshape\@subtitle
%% \fi
%% \makeatother}

%% \bigskip

%% door

%% \bigskip

%% %% Print the full name of the author.
%% \makeatletter
%% {\large\titlefont\bfseries\@firstname\ {\titleshape\@lastname}}
%% \makeatother

%% \end{center}

%% \bigskip
%% \bigskip

%% \begin{enumerate}

%% \item Stelling 1.
%% \item Stelling 2.
%% \item Stelling 3.
%% \item Stelling 4.
%% \item Stelling 5.
%% \item Stelling 6.
%% \item Stelling 7.
%% \item Stelling 8.
%% \item Stelling 9.
%% \item Stelling 10.

%% \end{enumerate}

%% \bigskip
%% \bigskip

%% %% Apart from the name and title of the supervisor, the following text is
%% %% dictated by the promotieregelement.
%% \begin{center}
%% Deze stellingen worden opponeerbaar en verdedigbaar geacht en zijn als zodanig goedgekeurd door de promotoren prof.\ dr.\ A.\ van Deursen and dr.\ A.\ Zaidman.
%% \end{center}

%% }

\end{document}

