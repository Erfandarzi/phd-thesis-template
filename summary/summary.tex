\chapter*{Summary}
\addcontentsline{toc}{chapter}{Summary}
\setheader{Summary}

Software developers today crave for feedback, be it from their peers in the form of code review,
static analysis tools like their compiler, or the local or remote execution of their tests in the
Continuous Integration (CI) environment. With the advent of social coding sites such as \github and
tight integration of CI services such as \travis, software development practices have fundamentally
changed.  Despite a highly alternated software engineering landscape, however, we still lack a
suitable holistic description of contemporary software development practices. Existing descriptions
such as the V-model are either too coarse-grained to describe an individual contributor's workflow, or
only regard a sub-part of the development process, like Test-Driven Development (TDD). In addition,
most existing models are \emph{pre-} rather than \emph{de-}scriptive.

By contrast, in this thesis, we perform a series of empirical studies to characterize the
individual constituents of Feedback-Driven Development (FDD): we study the prevalence and evolution
of Automatic Static Analysis Tools (ASATs), we explain the ``Last Line Effect,'' a phenomenon at
the boundary between ASATs and code review, we observe local testing patterns in the Integrated
Development Environment (IDE) of developers, compare them to remote testing on the CI server, and,
finally, should these quality assurance techniques have failed, we examine how developers debug
faults. We then compile this empirical evidence into a model of how today's software
developers work.

Our results show that developers employ the different techniques in FDD to best achieve their
current task in the most efficient way, often knowingly taking shortcuts to \emph{get the job
  done}. While this is efficient in the short term, it also bears risks, namely that prevention and
introspection activities fall short: developers might not configure or combine ASATs to their full
benefit, they might have wrong perceptions about the amount of time spent on quality-control,
quality-related activities such as testing could become an after-thought, and learning about debugging
techniques falls short. A relatively rigid, tool-enforced FDD process could help developers in not
committing some of these mistakes. Our thesis culminates in the finding that feedback loops are
the characterizing criterion of contemporary software development. Our model is flexible enough to
accommodate a broad band of modern workflows, despite large variances in how projects use and
configure parts of FDD.

\chapter*{Samenvatting}
\addcontentsline{toc}{chapter}{Samenvatting}
\setheader{Samenvatting}

{\selectlanguage{dutch}

Softwareontwikkelaars van vandaag hunkeren naar feedback over hun werk, danwel van hun peers via
code review, via statische analyse tools zoals hun compiler, ofwel via de uitvoering van testen, hetzij
lokaal of op afstand in de Continuous Integration (CI) omgeving. De strakke integratie van sociale
coding sites zoals \github en CI services zoals \travis hebben software ontwikkeling enorm
veranderd. Met deze grote verschuivingen op het vlak van software ontwikkeling missen we een
holistische beschrijving van hedendaagse software ontwikkelingspraktijken. Bestaande beschrijvingen
zoals het V-model zijn te grof om een individuele workflow te beschrijven of gaan alleen over een
onderdeel van het ontwikkelingsproces, zoals Test-Driven Development (TDD). Bovendien zijn de
bestaande modellen meer \emph{pre-} dan \emph{de-}scriptief.

In deze thesis daarentegen doen we een reeks empirische studies om de individuele onderdelen van
Feedback-Driven Development te beschrijven: we onderzoeken hoe wijdverspreid het gebruik van
Automatic Static Analysis Tools (ASATs) is, bekijken de evolutie van hun
gebruik en we leggen
het ``Last Line Effect'' uit, een fenomeen op het snijvlak van ASATs en code
reviews. Ook observeren we
de lokale testpatronen van ontwikkelaars in hun Integrated Development
Environment en vergelijken we
die lokale patronen met het op afstand testen op de CI server. Vervolgens bestuderen we hoe
ontwikkelaars fouten debuggen in het geval dat de voorgaande maatregelen om de kwaliteit te bewaken
falen. Ten slotte verzamelen we het empirische bewijs dat we hebben verkregen om
tot een model te komen van hoe softwareontwikkelaars heden ten dage werken.


Onze resultaten tonen dat programmeurs de verschillende technieken in FDD gebruiken om hun
programmeeropdracht op de meest effici{\"e}nte manier uit te voeren, waarbij ze vaak bewust een
shortcut nemen om de klus te klaren. Het valt niet te ontkennen dat die op korte termijn effici{\"e}nt
is, maar deze manier van werken brengt ook risico’s met zich mee, vooral op het vlak van preventie
en introspectie-activiteiten die te kort schieten. Zo kan het voorkomen dat programmeurs hun ASATs
niet optimaal configureren of combineren, ze een verkeerde perceptie hebben qua tijdsbesteding van
kwaliteitscontrole, ze activiteiten verwant aan kwaliteitsbewaking, zoals testen, als
bijkomstigheid beschouwen en zichzelf onvoldoende scholen op het gebied van debuggingtechnieken. Een relatief rigide, door tools gehandhaafd FDD proces kan ontwikkelaars begeleiden om
deze fouten niet te maken. Onze thesis culmineert in de vondst dat feedbacklussen het
karakteriserende criterium zijn van moderne softwareontwikkeling. Ons model is flexibel genoeg om
er een brede waaier aan moderne workflows in onder te brengen, ondanks de grote variatie in hoe
projecten delen van FDD gebruiken en configureren.}



