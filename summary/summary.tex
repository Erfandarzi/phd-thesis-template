\chapter*{Summary}
\addcontentsline{toc}{chapter}{Summary}
\setheader{Summary}

Software developers today crave for feedback, be it from their peers in the form of code review,
static analysis tools like their compiler, or the local or remote execution of their tests in the
Continuous Integration (CI) environment. With the advent of social coding sites like \github and
tight integration of CI services like \travis, software development practices have fundamentally
changed.  Despite a highly alternated software engineering landscape, however, we still lack a
suitable holistic description of contemporary software development practices. Existing descriptions
like the V-model are either too coarse-grained to describe an individual contributor's workflow, or
only regard a sub-part of the development process, like Test-Driven Development (TDD). In addition,
most existing models are \emph{pre-} rather than \emph{de-}scriptive.

By contrast, in this thesis, we perform a series of empirical studies to characterize the
individual constituents of Feedback-Driven Development (FDD): we study the prevalence and evolution
of Automatic Static Analysis Tools (ASATs), we explain the ``Last Line Effect,'' a phenomenon at
the boundary between ASATs and code review, we observe local testing patterns in the Integrated
Development Environment (IDE) of developers, compare them to remote testing on the CI server, and,
finally, should these quality assurance techniques have failed, we examine how developers debug
faults. We then compile this empirical evidence into an initial theory of how today's software
developers works.

Our results show that developers employ the different techniques in FDD to best achieve their
current task in the most efficient way, often knowingly taking shortcuts to \emph{get the job
  done}. While this is efficient in the short term, it also bears risks, namely that prevention and
introspection activities fall short: developers might not configure or combine ASATs to their full
benefit, they might have wrong perceptions about the amount of time spent on quality-control,
quality-related activities like testing could become an after-thought, and learning about debugging
techniques falls short. A relatively rigid, tool-enforced FDD process could help developers in not
committing some of these mistakes. Our thesis culminates in the finding that feedback loops are
the characterizing criterion of contemporary software development. Our model is flexible enough to
accommodate a broad band of modern workflows, despite large variances in how projects use and
configure parts of FDD.

\chapter*{Samenvatting}
\addcontentsline{toc}{chapter}{Samenvatting}
\setheader{Samenvatting}

{\selectlanguage{dutch}

Software ontwikkelaars van vandaag hunkeren naar feedback over hun werk, zij het van hun peers qua
code review, static analysis tools zoals hun compiler, of de lokale of remote executie van hun
tests in de Continuous Integration (CI) omgeving. Door sociaal coding sites zoals \github en een
strakke integratie van CI services zoals \travis zijn software ontwikkeling praktijken vandaag
enorm verandert. Ondanks deze grote veranderingen hebben nog steeds geen passende holistische
beschrijving van tijdgenoot software ontwikkelings- praktijken. Bestaande beschrijvingen zoals de
v-model zijn te grof om een individueel workflow te beschrijven of gaan alleen over een onderdeel
van de ontwikkeling process, zoals Test-Driven Development (TDD). Daarnaast zijn de bestaande
modellen meer \emph{pre-} als \emph{de}-scriptief.

Daarentegen doen we een series van empirische studies in deze thesis om de individuele
bestanddelen van Feedback-Driven Development te karakteriseren: we onderzoeken de verbreidheid en
evolutie van Automatic Static Analysis Tools (ASATs), we leggen de ``Last Line Effect'' uit, een
fenomeen aan de grens tussen ASATs en code review, we volgen lokale test monsteren in de Integrated
Development Environment (IDE) van ontwikkelaars, vergelijken hun met remote testen op de CI server,
en, uiteindelijk, als deze kwaliteitsverzekeringsmethoden mislukt zijn, onderzoeken we hoe
ontwikkelaars foutjes debuggen. Vervolgens compileren we deze empirische evidentie in een initiale
theorie over hoe de software ontwikkelaars van vandaag werken.

Onze resultaten tonen dat programmeurs de verschillende technieken in FDD gebruiken om hun actueel
opdracht op de meest effici{\"e}nte manier uit te voeren, waartoe ze vaak bewust de weg afkorten
\emph{to get the job done.} Terwijl dit effici{\"e}nt is op korte termijn, draagt het ook risico's,
vooral dat preventie en introspectie activiteiten te kort kunnen komen: programmeurs configureren
hun ASATs wellicht niet voor hun maximaal voordeel, ze hebben misschien een verkeerde perceptie van
de tijd die ze op kwaliteitskontrolle gebruiken, kwaliteits-verwante activiteiten zoals testen
worden misschien een afterthought, een leren over debugging technieken komt te kort. Een best wel
starre, door een tool doorgezet FDD proces kan ontwikkelaars dus helpen deze foutjes niet te
maken. Onze thesis culmineert in de vondst dat feedback lussen de karakteriserende criterium van
moderne software ontwikkeling zijn. Onze model is flexibel genoeg om een brede band aan moderne
workflows onder een dak te brengen, ondanks grote variaties in hoe projecten delen van FDD
gebruiken en configureren.}



